\documentclass{beamer}
\usepackage[utf8]{inputenc} 
\title{Poppy 3B}
\author{Adam Rembiewski}
\institute{UWM}
\date{\today}
\usepackage{amsfonts}
\usepackage[MeX]{polski}
\usepackage{hyperref}
\hypersetup{colorlinks=true,
    linkcolor=blue,
    filecolor=magenta,      
    urlcolor=cyan,}
\usetheme{Warsaw}
\begin{document}
\frame{\titlepage}



\section{Czym jest Poppy 3B}
\begin{frame}{Czym tak naprawde jest Poppy 3B}
\begin{itemize}
\item{Jest to amerykański satelita wywiadu elektronicznego serii Poppy. Pozostawał utajniony do września 2005. Zbudowany przez Naval Research Laboratory, NRL.}
\pause
\item{Wystrzelony na orbitę 11 stycznia 1964 razem z Poppy 3A, Poppy 3C, SECOR 1B i FTV 2354.Zbierał dane o parametrach pracy radzieckich radarów obrony powietrznej i antybalistycznej.}
\pause
\item{Statek pozostaje na orbicie okołoziemskiej, której trwałość szacuje się na 1000 lat.}
\end{itemize}
\end{frame}

\section{Nazewnictwo}
\begin{frame}{Skąd nazwa Poppy 3B}
\begin{itemize}
\item{Satelity serii Poppy i jej poprzednika – serii GRAB – miały wiele nazw, które miały zataić prawdziwe ich przeznaczenie.}
\pause
\item{Program pierwotnie nazwany był Plotkarzem, a później GRAB. By przeznaczenie satelitów nie było jasne dla ZSRR, projekt nazywano też GREB.}
\pause
\item{By jeszcze bardziej zaciemnić przeznaczenie satelitów serii GRAB i Poppy, wysyłano je pod nazwą Solrad. Miało to wskazywać, że będą one prowadzić obserwacje Słońca.}
\end{itemize}
\end{frame}

\section{Bibliografia}
\begin{frame}{Bibliografia}
\begin{itemize}
\item {Gunter Krebs: Poppy (20 in series) (ang.). Gunter's Space Page. [dostęp 2013-09-30].}
\item {Jonathan McDowell: Launchlog (ang.). Jonathan's Space Home Page. [dostęp 2013-09-30].}
\item {1964-001E (ang.). W: NSSDCA Master Catalog [on-line]. NASA. [dostęp 2017-06-27].}
\item {Space 40 (cz.)}
\item {Mark Wade: Poppy (ang.). W: Encyclopedia Astronautica [on-line]. [dostęp 2017-06-27].}
\end{itemize}
\end{frame}

\section{Linki zewnętrzne}
\begin{frame}{Linki zewnętrzne}
\begin{itemize}
\item{Robert A. McDonald, Sharon K. Moreno: \href{http://www.nro.gov/history/csnr/programs/docs/prog-hist-03.pdf}{Raising the Periscope... Grab and Poppy: America's Early ELINT Satellites} (ang.). NRO, wrzesień 2005. [dostęp 2013-09-29].}
\item{\href{http://www.nro.gov/foia/} {History of The Poppy Satellite System} (ang.). NRO. [dostęp 2013-09-28].}
\item{\href{http://www.nro.gov/foia/docs/U.S.\%20Navy-NRO\%20Program\%20C\%20Electronic\%20Intelligence\%20Satellites\%20\%281958-1977\%29.pdf}{U.S. Navy/NRO Program C Electronic Intelligence Satellites (1958-1977)} (ang.). NRO. [dostęp 2013-09-28].}
\end{itemize}
\end{frame}

\begin{frame}
\frametitle{Spis Treści}
\tableofcontents
\end{frame}


\end{document}
